\section{Conclusion}
Finding a \textbf{conclusion}, we can say that science and reality are not always coming together
because politics have to be aware of a lot of different interests \cite{NYT1} \cite[1]{ChinaEvol}. 
The given examples displayed off clearly that a switch to a 100 percent sustainable energy system is
possible, not only technically but also in an economic, profitable way (Section \ref{sec:Sweden}); (Section \ref{sec:NY}); (Section \ref{sec:China}).
Very accurate plans and schedules on different scales, be that for countries, states or cities have been worked out already.
The consequences of neglecting these strategies are shared with the world community \cite{Bible}.
\par
Coming back to my given scenario of 25 years with oil left, the prognosis are showing off clearly that the major part of the world economy
will still run on fossil resources at this time \cite[633]{UN_Climate_Goals} , which would in fact cause a vast collapse of the world economy \cite{Bible} \cite[633]{UN_Climate_Goals}.
\par
We have to admit, that this scenario is limiting the usable resources to the amount, compatible with the UN 2.0 degree Celsius goals \cite[14]{BP} \cite{Industries}.
This agreement could be easily ignored, putting up with the climate effects, or using Carbon Caption Systems, as the UN proposes \cite{UN_Plan}, to save the reliability of the economic system.
As mentioned in section (Section \ref{sec:prognosis}), pressured by the climate change caused damage, politics would most likely decide to install nuclear reactors
to prevent as well natural and economic catastrophes.
\par
{\Large My conclusion, then, is that}, although we are not using all of our technical possibilities, risking unnecessarily high climate change \cite{Sweden}, caused damages,
it is unlikely to undergo an economic collapse because of oil lacks.
Major parts of national governments do not prepare for long term oil lacks by now, with a few exceptions, for instance Sweden \cite{Sweden}, because there is no urgency \cite[4]{BP} \cite{UN_Plan}.
Nuclear reactors are providing the possibility of fast reaction, offering a reliable energy supply \cite{nuclear}.
Dearer, faulty for nature and human beings, but possibly, oil could be produced by new techniques out of sand or seabed \cite{canadianSand}, stabilizing the still oil-dependent industry sectors for a long period of time\cite{IEAApril}.
This new cost-provided urgency could force leaders of the world, to finally start a more radical \textbf{process} of implementing a sustainable, fossil-free energy system, if there is no vast technological development, solving the problem earlier \cite{UN_Plan} \cite{ChinaEvol} \cite{WEO}.
