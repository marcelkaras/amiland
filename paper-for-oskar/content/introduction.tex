%https://libanswers.snhu.edu/faq/83432
%https://aus.libguides.com/apa/apa-no-author-date
\section{Introduction}% \textit{1st draft}} 
\label{sec:introduction}
I am Marcel, a German student, participating the Intensive English Program at Western Washington University for one quarter
with the \textbf{purpose} of raising my English level to be prepared for coming international communication and collaboration in 
academic and businesslike settings.
To write this research paper is a task given by my grammar teacher, Oskar Norlander, with the expectation to find a fitting 
question, and answer it appropriately using phrases, vocabulary, and methods I learned previously.
My home country is Germany, where, while I am studying physics at the Technical Univerity Dortmund, my interests in science and nature
are widespread. I have chosen the question, 
\par
"What plans and ideas exist in the world community to prevent an economic 
disaster when oil resources dry up in 25 years \cite[14]{BP} \cite{Industries}? Which organizations 
might have worked out the best masterplan, and is this even enough?"
\par
I am passionate about the topic of renewable energies and, in addition, I  got the opportunity to get an in-depth view
of actual energy research and politics during my one-week internship at the International Energy Agency Paris, which has provoked even
more curiosity in me. 
I have learned about the technical difficulties there are when we try to integrate renewable power supplies in the existing grid,
having a high requirement of reliability and stable frequencies \cite{inertia}.
\par
The anxiety, losing the supply of fossil resources might by high, according  to an interview I had conducted with Sylvia Bayer, the head oil market reporter of the International Energy Agency \cite{IEAApril}.
So the OECD (Organisation for Economic Co-operation and Development) states are holding huge amounts of storage for the case of interrupted oil supply \cite[42]{IEAApril}. 
These storages might be the solution for a few months without external oil, but it increases fear in me, 
thinking about a scenario of worldwide, long time oil shortage, in a, as I would like to show in the next chapter, society running on fossil fuels.
Whereas, scientists \cite{NY_Jacobson} found solutions to limit disruptions in the economy's energy supply, in the case of fossil resources we are using for our demands today decisively,
decisionmakers might not be looking forwards as foresighted.
In the end, it is not up to what we could have done to avoid such an event of "biblical proportions," as Calcuttawala describes a worldwide blackout\cite{Bible}, but what we have done.
The economic damage might be tremendous, today's life a "thing of the past," and the supply of basic needs in serious danger \cite{Bible}.
{\Large In other words}, do national governments understand the urgency of these problems and are they \textbf{processing} a plan to implement solutions early enough to prevent a total blackout? 
\par
Incipient, I am intent to specify our dependence on fossil energy sources to estimate a scale and to elaborate the current \textbf{setting} of the worldwide energy market.
The UN set basic goals to reduce CO2 production in order to slow the climate change. 
This goal is directly connected with cutting oil consumption, the main factor of man-produced CO2 \cite{OILCombustion}.
So I will try to clarify UN goals and check how well the nations are doing to reach them.
To consider example countries representing contrary regions and political background, I would like to
focus on China, Sweden, and the City of New York, \textbf{analyzing} their energy strategies for the coming years.
Traditional participants of the energy market, for instance oil companies, are expected to have a different view on this subject in order to save their business \cite{capitalism}.
Their view I would like to elaborate in section \ref{sec:BP}.
Balancing out how well statements of the different parties might fit in other`s \textbf{perspectives}, we will be able to point out two different answers and their overlap.
