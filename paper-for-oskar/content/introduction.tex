\section{Introduction}% \textit{1st draft}} 
\label{sec:introduction}
I am Marcel, a german student participating the Intensive English Program at Western Washington University for one quarter
with the purpose of raising my English level to be prepared for coming international communication and collaboration in 
academic and businesslike settings.
To write this research paper is a task given by my grammar teacher Oskar Norlander with the expectation to choose a fitting 
question and answer it appropriately using phrases, vocabulary and methods I learned precedent.
My home country is Germany, where, while I am studying physics at TU Dortmund, my interests about science and nature
are widespread. I have chosen the question, \par
"What plans and ideas exist in the world community to prevent an economic 
disaster when oil resources dry up in eight years? Which organizations 
might have worked out the best masterplan, and is this even enough?". \par
I am panting for the topic of renewable energies and, in addition, I have got the opportunity to get an in-depth view
of actual energy research and politics while my one-week internship at the International Energy Agency Paris, which has awake even
more curiosity in me. 
I already learned about the technical difficulties there are when we try to integrate renewable power supplies in the existing grid,
having a high requirement of reliability and not at least stable frequencies.
Whereas the scientists found solutions to limit lacks in our energy supply, in the case of fossil resources we are using for our demands today decisively,
politics might not looking forwards as predictive.
In the end it is not up to, what we could have done to avoid such a catastrophic event of a worldwide blackout, but what we have done.
The economic damage might be uncountable, today's life unimaginable, and the supply of basic needs in serious danger.\par
In other words, do the politics understand the urgency of these problems and pointed out a plan to implement solutions early enough to prevent a total blackout? \par
I am first intend to specify our dependence od fossil energy sources to estimate a scale.
The UN set basic goals according to reduce CO2 production, in order to slow the climate change. 
This goal is directly connected with cutting oil consumption, the main factor of man-produced CO2.
So I try to valid the UN goals and check how well the nations are doing to reach them.
To instance example countries representing contrary regions and political background I would like to
focus on China, Sweden, and the City of New York, presenting their energy strategies for the coming years.
As we will see there are some forces working against innovation within the energy market. We will balance 
out how big their impact on decisionmaking might be.
%\textit{Cites will be added during the research project}



